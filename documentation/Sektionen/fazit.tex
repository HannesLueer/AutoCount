\selectlanguage{ngerman}
\section{Fazit}\label{ch:Fazit}
% Zukünftig:
% - HTTPS
% - Anwendung für Datenbank für Mitarbeiter
% - Auswahl der Parkplätze in App; andere Darstellung in App, bei großflächigem Einsatz beispielsweise auf Karte
% - selbes System für die Mitarbeiterparkplätze --> diese vom Zähler abziehen
% - separates Zählsystem für Mitarbeiter anbieten

Die Anwendung und der Test des entwickelten Fahrzeugzählsystems wurden in einem realen Parkhaus durchgeführt, wobei sich das System als zuverlässig in der Zählung der Fahrzeuge erwies.
Der Vergleich der beiden implementierten Zählverfahren ergab, dass beide Verfahren eine höhere Genauigkeit und Robustheit aufweisen, Verfahren 2 jedoch effizienter ist.
Die Ergebnisse dieser Tests sind vielversprechend und zeigen das Potenzial des Systems für den praktischen Einsatz in der Parkhausauslastungsmessung auf.

Für eine Weiterentwicklung des Systems wurden folgende Aspekte identifiziert.

Um die Performance des Systems zu erhöhen, sollten Alternativen für den Mini-Computer in Betracht gezogen werden.
Der Raspberry Pi 3 B+ hat sich für diese Aufgabe als nicht leistungsstark genug herausgestellt.
Vertreter der Jetson-Reihe von NVIDIA oder Single-Board-Computer der ASUS Tinker-Board-Reihe bieten beispielsweise eine deutlich höhere Rechenleistung.
Gleichzeitig werden alle anderen Anforderungen für die Objekterekennung und die Kommunikation zum Server von den genannten Boards erfüllt.

Um die Sicherheit und den Datenschutz zu gewährleisten, sollte eine Umstellung auf HTTPS erfolgen.
Dies würde die Übertragung der Daten verschlüsseln und die Vertraulichkeit der Informationen gewährleisten.
Dadurch wird sichergestellt, dass die erfassten Messdaten und vor allem Zugangsdaten vor unbefugtem Zugriff geschützt sind.

Im aktuellen Prototypen müssen Parkhäuser initial manuell in der Datenbank angelegt werden.
Eine für den großflächigen Einsatz nötige Erweiterung wäre Implementierung eines Systems, über welches Mitarbeiter bequem und fehlerfrei neue Stationen anlegen können.

Eine Weiterentwicklung der Anwendung könnte eine dynamische Parkplatzauswahl in der App sein.
Statt fester Standortzuweisungen könnten die verfügbaren Parkplätze und Parkhäuser in der App angezeigt werden, wobei die Darstellung bei großflächigem Einsatz auf einer Karte erfolgen könnte.
Dies würde den Benutzern ermöglichen, freie Parkplätze leichter zu finden und die Effizienz des Parkplatzmanagements zu steigern.
Außerdem wäre so nicht für jedes Parkhaus eine eigene App nötig.

Die Einbindung der Mitarbeiterparkplätze in das Zählsystem wäre eine sinnvolle Erweiterung.
In der Umsetzung des Prototyps wird nicht beachtet, dass innerhalb des Parkhauses ein separater Bereich für Mitarbeiter der Hochschule existiert.
Durch Anbringung eines zweiten Sensors und Subtrahieren der Zählergebnisse der Mitarbeiterparkplätze von den Gesamtzählergebnissen kann auch eine Differenzierung zwischen den Bereichen vorgenommen werden.

Zusammenfassend eröffnet die entwickelte Fahrzeugzählungsanwendung eine Vielzahl von Möglichkeiten für zukünftige Verbesserungen und Erweiterungen.
Die Kombination dieser Aspekte könnte das Fahrzeugzählsystem zu einem sinnvollen Werkzeug für ein effizientes und intelligentes Parkplatzmanagement machen.
Bereits der Prototyp konnte zeigen, wie ohne den Einsatz spezieller Hardware eine Zählung ermöglicht wird.
