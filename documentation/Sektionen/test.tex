\selectlanguage{ngerman}
\section{Anwendung und Test}\label{ch:Test}
% - Installation in Parkhaus und Aufzeichnung
%   - Zu Testzwecken separat von Erkennung
% - Allgemeine Erklärung zur Nutzung des Systems
% - Vergleich der zwei Zählverfahren

In diesem Kapitel wird die Anwendung und der Test des entwickelten Systems vorgestellt.
Zunächst wird die allgemeine Nutzung des Systems erläutert.
Anschließend werden die Installation des Systems im Hochschul-Parkhaus und die Aufzeichnung von Testdaten beschrieben sowie ein Vergleich der beiden entwickelten Zählverfahren vorgenommen.

\subsection{Nutzung des Systems}
Das entwickelte System zur Fahrzeugzählung bietet eine einfache und benutzerfreundliche Anwendung.
Für den Betreiber des Parkhauses ist es lediglich nötig die Sensoreinheit an der Ein- und Ausfahrt des Parkhauses zu montieren und für Strom und eine Internetverbindung zu sorgen.
Nachdem die Authentifizierungsdaten hinterlegt wurden, kann direkt mit der Zählung begonnen werden.
Auf der Seite des Nutzers wird aufgrund der plattformübergreifenden App eine Installation und Nutzung auf verschiedenen Endgeräten, darunter Android, Windows, iOS und macOS, ermöglicht.
Die Zählergebnisse werden dabei in Echtzeit angezeigt und können überwacht werden.

\subsection{Installation und Testaufzeichnung}
Zur Überprüfung der Genauigkeit des entwickelten Systems wurde ein Test durchgeführt.
Das System wurde dabei im Parkhaus der Hochschule Coburg installiert, um statt der eigentlichen live Fahrzeugerkennung zunächst ein Video aufzunehmen, welches als Vergleichsmaterial der Verfahren genutzt werden kann.
Auch, weil zum aktuellen Zeitpunkt im Parkhaus keine Internetverbindung besteht, wurden alle weiteren Tests auf Basis des aufgezeichneten Videomaterials durchgeführt.
Technisch ergibt sich hierdurch kein Unterschied im Vergleich zur Echtzeitanalyse, da die verwendeten Verfahren jeweils nur auf den aktuellen Frame zurückgreifen.

\subsection{Vergleich der beiden Zählverfahren}
Im Rahmen der Arbeit wurden zwei verschiedene Zählverfahren (siehe Kapitel~\ref{ch:Umsetzung_Sensor}) entwickelt und implementiert.
Um zu entscheiden, welches dieser Verfahren besser ist, wurden diese auf Korrektheit und Effizienz verglichen.
% Korrektheit:
Hierzu wurde ein 10 Minuten langer Ausschnitt der Aufzeichnung von beiden Verfahren analysiert und mit der manuellen Zählung der Autos verglichen.
Dabei ergaben sich folgende Werte.

\begin{table}[h]
	\centering
	\begin{tblr}{
			colspec={lrrr},
			row{even}={bg=gray!5},
			row{odd}={bg=gray!20},
			row{1}={bg=black!80,fg=white},
		}
		\centering
		Verfahren                                & Autos rein & Autos raus & Autos gesamt \\
		\hline
		Manuelle Zählung                         & 10         & 4          & 6            \\
		Verfahren 1 (Kapitel~\ref{ch:Sensor_v1}) & 10         & 4          & 6            \\
		Verfahren 2 (Kapitel~\ref{ch:Sensor_v2}) & 10         & 4          & 6            \\
	\end{tblr}
	\caption{Ergebnisse des Tests}\label{tab:TestErgebnisse}
\end{table}

Demnach erzielen beide Verfahren korrekte Ergebnisse.
Während des Testvideos fuhren außerdem drei Autos durch den Bildausschnitt, die nicht ins Parkhaus fuhren oder aus dem Parkhaus kamen.
Diese wurden jeweils korrekterweise von den Verfahren nicht beachtet.

% Effizienz:
Auch die Effizienz der Verfahren wurde verglichen.
Hierzu wurde die Zeit ermittelt, die jedes Verfahren pro Frame zur Verarbeitung benötigt.
Die Ergebnisse des 10-minütigen Tests wurden als Boxplots in Abbildung~\ref{fig:frameTimes_Boxplots} dargestellt.
Aufgrund des ursprünglich geplanten Einsatzes eines Raspberry Pi, wurde der Test auf diesem durchgeführt.
Hierbei ergab sich ein Median für Verfahren 1 von rund 2,04 Sekunden und für Verfahren 2 von rund 1,85 Sekunden.
Als Boxplot sind die Messwerte in Abbildung~\ref{fig:frameTimes_Boxplot_pi} zu sehen, in der aufgrund der Ausreißer deutlich wird, dass Verfahren 2 in seltenen Fällen sehr lange für die Verarbeitung eines einzelnen Frames benötigt.
Dies macht sich bereits im Mittelwert bemerkbar, da dieser bei rund 1,92 liegt, während er für Verfahren 1 dem Median entspricht.

\begin{figure}[h]
	\myImagePos{}
	\subfloat[Raspberry Pi (4~Threads @ 1,4~GHz)]{\label{fig:frameTimes_Boxplot_pi}
		\includesvg[inkscapelatex=false,width=0.49\myImageWidth]{Bilder/frame_times_pi.svg}}
	\hfill
	\subfloat[PC mit i5-8259U (8~Threads @ 2,3~GHz)]{\label{fig:frameTimes_Boxplots_pc}
		\includesvg[inkscapelatex=false,width=0.49\myImageWidth]{Bilder/frame_times_pc.svg}}
	\caption[Darstellung der Boxplots der benötigten Zeit pro Frame für die zwei entwickelten Verfahren]{Darstellung der Boxplots der benötigten Zeit pro Frame für die zwei entwickelten Verfahren auf dem Raspberry Pi~3~B+ und einem Intel NUC NUC8i5BEH (Quelle: eigene Darstellung)}\label{fig:frameTimes_Boxplots}
\end{figure}

Um sicherzugehen, dass die starken Ausreißer nicht an äußeren Umständen des Raspberry Pi, wie beispielsweise dessen Stromversorgung oder passiven Kühlung, liegen, wurde der Test ebenfalls auf einem Rechner mit aktiver Kühlung, stabiler Stromversorgung und mehr Rechenleistung durchgeführt.
Erwartungsgemäß konnten allgemein bessere Ergebnisse erzielt werden.
Die starken Ausreißer bei Verfahren 2 blieben allerdings bestehen.
Im Median erreichte jetzt Verfahren 1 mit rund 0,10 Sekunden sogar einen minimal besseren Wert als Verfahren 2 mit rund 0,11 Sekunden.

Aus den Messdaten wird also ersichtlich, dass Verfahren 2 auf dem Raspberry Pi sowohl im Median als auch im Mittelwert performanter ist als Verfahren 1.
Allerdings ist Verfahren 1 im Mittel ähnlich effizient aber besitzt vor allem eine niedrigere Varianz.
Daher wird eine konstantere Zeit pro Frame ermöglicht, was die Messung berechenbar macht.
Aus diesen Gesichtspunkten wird sich für Variante 1 entschieden.

Da Verfahren 2 die Autos anhand ihrer Merkmale erkennt, um diesen IDs zuzuweisen, müssen Informationen zu diesen gespeichert werden.
Daher ist die Vermutung, dass intern Datenstrukturen aufgebaut oder umstrukturiert werden, was dazu führt, dass die Hardware zusätzlich zur eigentlichen Berechnung kurzzeitig stärker beansprucht wird und die Berechnung des Frames infolgedessen deutlich mehr Zeit benötigt.

