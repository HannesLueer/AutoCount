\selectlanguage{ngerman}
\section{Anwendung und Test}\label{ch:Test}
% - Installation in Parkhaus und Aufzeichnung
%   - Zu Testzwecken separat von Erkennung
% - Allgemeine Erklärung zur Nutzung des Systems
% - Vergleich der zwei Zählverfahren

In diesem Kapitel wird die Anwendung und der Test des entwickelten Systems vorgestellt.
Zunächst wird die allgemeine Nutzung des Systems erläutert.
Anschließend werden die Installation des Systems im Hochschul-Parkhaus und die Aufzeichnung von Testdaten beschrieben, sowie ein Vergleich der beiden entwickelten Zählverfahren vorgenommen.

\subsection{Nutzung des Systems}
Das entwickelte System zur Fahrzeugzählung bietet eine einfache und benutzerfreundliche Anwendung.
Für den Betreiber des Parkhauses ist es lediglich nötig die Sensoreinheit an der Ein- und Ausfahrt des Parkhauses zu montieren und vor Strom und eine Internetverbindung zu sorgen.
Nachdem die Authentifizierungsdaten hinterlegt wurden, kann direkt mit der Zählung begonnen werden.
Auf der Seite des Nutzers wird aufgrund der plattformübergreifenden App eine Installation und Nutzung auf verschiedenen Endgeräten, darunter Android, Windows, iOS und macOS, ermöglicht.
Die Zählergebnisse werden dabei in Echtzeit angezeigt und können überwacht werden.

\subsection{Installation und Testaufzeichnung}
Zur Überprüfung der Genauigkeit des entwickelten Systems wurde ein Test durchgeführt.
Das System wurde dabei im Parkhaus der Hochschule Coburg installiert, um statt der eigentlichen live Fahrzeugerkennung zunächst ein Video aufzunehmen, welches als Vergleichsmaterial der Verfahren genutzt werden kann.
Auch, weil zum aktuellen Zeitpunkt im Parkhaus keine Internetverbindung besteht, wurden alle weiteren Tests auf Basis des aufgezeichneten Videomaterials durchgeführt.
Technisch ergibt sich hierdurch kein Unterschied im Vergleich zur Echtzeitanalyse, da die verwendeten Verfahren jeweils nur auf den aktuellen Frame zurückgreifen.

\subsection{Vergleich der beiden Zählverfahren}
Im Rahmen der Arbeit wurden zwei verschiedene Zählverfahren (siehe Kapitel~\ref{ch:Umsetzung_Sensor}) entwickelt und implementiert.
Um zu entscheiden, welches dieser Verfahren besser ist, wurden diese auf Korrektheit und Effizienz verglichen.
% Korrektheit:
Hierzu wurde ein 10 Minuten langer Ausschnitt der Aufzeichnung von beiden Verfahren analysiert und mit der manuellen Zählung der Autos verglichen.
Dabei ergaben sich folgende Werte.

\begin{table}[h]
	\centering
	\begin{tblr}{
			colspec={lrrr},
			row{even}={bg=gray!5},
			row{odd}={bg=gray!20},
			row{1}={bg=black!80,fg=white},
		}
		\centering
		Verfahren                                & Autos rein & Autos raus & Autos gesamt \\
		\hline
		Manuelle Zählung                         & 10         & 4          & 6            \\
		Verfahren 1 (Kapitel~\ref{ch:Sensor_v1}) & 10         & 4          & 6            \\
		Verfahren 2 (Kapitel~\ref{ch:Sensor_v2}) & 10         & 4          & 6            \\
	\end{tblr}
	\caption{Ergebnisse des Tests}\label{tab:TestErgebnisse}
\end{table}

Demnach erzielen beide Verfahren korrekte Ergebnisse.
Während des Testvideos fuhren außerdem drei Autos durch den Bildausschnitt, die nicht ins Parkhaus fuhren oder aus dem Parkhaus kamen.
Diese wurden jeweils korrekterweise von den Verfahren nicht beachtet.

% Effizienz:
Auch die Effizienz der Verfahren wurde verglichen.
Hierzu wurde die Zeit ermittelt, die jedes Verfahren pro Frame zur Verarbeitung benötigt.
Die Ergebnisse des 10-minütigen Tests wurden als Boxplot in Abbildung~\ref{fig:frameTimes_Boxplots} dargestellt.
Dabei ist ersichtlich, dass Verfahren 2 effizienter ist und daher weniger Zeit pro Frame benötigt.
Auch ist die Varianz bei Variante 1 höher.
Aus diesen Gesichtspunkten wird sich für Variante 2 entschieden.

\begin{figure}[h]
	\myImagePos{}
	\includesvg[inkscapelatex=false,width=\myImageWidth]{Bilder/frame_times_pc.svg}
	\caption[Darstellung der Boxplots der benötigten Zeit pro Frame für die zwei entwickelten Verfahren]{Darstellung der Boxplots der benötigten Zeit pro Frame für die zwei entwickelten Verfahren (Quelle: eigene Darstellung)}
	\label{fig:frameTimes_Boxplots}
\end{figure}
