\selectlanguage{ngerman}
\section{Idee}\label{ch:Einleitung}
% - Problemstellung: aktuelle Situation 
% - Abwägung verschiedener Möglichkeiten (Lichtschranke, Näherungssensor für jeden Parkplatz, Dreh-Sensor an Schranke) und Probleme dieser (Menschen und Radfahrer könnten auch erkannt werden, Bewegungsrichtung unklar, wenn auf falscher Seite gefahren wird, Näherungssensoren für jeden Parkplatz sehr teuer)
% - Entschluss: Kamera-basierte Lösung notwendig 
Das Parkhaus an der Hochschule Coburg, welches direkt am Campus positioniert ist, umfasst 530 Parkplätze für Studierende und Mitarbeiter \cite{parkhaus}.
Dabei sind die Parkmöglichkeiten für Studenten und Mitarbeiter durch eine Schranke getrennt.
Die Haupteinfahrt ist jedoch für beide Parkflächen dieselbe.
Es gibt außerdem einen Parkplatz in der Sonneberger Straße \cite{anfahrt}.
Dieser ist jedoch aufgrund der Lage für Autofahrer unattraktiver als das Parkhaus.
Im vergangenen Wintersemester (2022/23) gab es 4890 immatrikulierte Studenten und 518 Mitarbeiter \cite{HSCoburgZahlen}.

\begin{figure}[h]
	\myImagePos{}
	\includegraphics[width=0.6\myImageWidth]{Bilder/parkhaus_HSCoburg.jpg}
	\caption[Parkhaus der Hochschule Coburg]{Parkhaus der Hochschule Coburg (Campus Friedrich Streib) (Quelle: \cite{parkhaus})}
	\label{fig:Parkhaus}
\end{figure}

Wenn man als Student eine Vorlesung zu den Stoßzeiten besuchen möchte und auf das Auto angewiesen ist, stellt man häufig fest, dass die Auslastungs-Anzeige des Parkhauses „BESETZT“ darstellt.
Das Problem dieser Anzeige ist, dass man sich nie sicher sein kann, was diese Anzeige nun genau darstellt.
Ist gar kein Parkplatz mehr frei, sind nur noch wenige frei oder wird sogar eine Fehlinformation angezeigt?
Häufig lassen sich trotz der Anzeige, dass das Parkhaus besetzt wäre, noch freie Parkplätze finden.

Im Rahmen des Moduls „Hardware cyber-physischer Systeme“ soll eine Möglichkeit gefunden werden, eine exakte Anzeige der Parkhausbelegung umzusetzen.
Dafür soll --- wie in größeren Parkhäusern üblich --- dargestellt werden, wie viele Parkplätze noch frei sind.
Hierfür sind verschiedene Ansätze möglich.
Zum einen wäre eine Umsetzung denkbar, bei welcher am Eingang des Parkhauses eine Lichtschranke installiert wird.
Diese soll jeweils Autos zählen, welche sich über die Einfahrt in das Parkhaus hinein- bzw.\ durch die Ausfahrt hinausbewegen.
Ein großes Problem bei diesem Ansatz ist, dass das System potenziell jedes Objekt oder jede Person detektieren würde.
Somit wäre die Anzeige der Parkhausbelegung fehlerhaft.
Außerdem wäre eine Umsetzung denkbar, bei welcher auf jedem Parkplatz ein Näherungssensor angebracht wird, welcher erfasst, ob sich auf dem jeweiligen Parkplatz ein Objekt befindet oder nicht.
Dieses Konzept würde vermutlich sehr gut funktionieren, bringt jedoch einen hohen Kostenaufwand mit sich, weswegen diese Variante verworfen wird.

Als potenziell beste Lösung hat sich eine Kamera-basierte Umsetzung herausgestellt.
Bei dieser Idee wird eine Kamera an der Einfahrt des Parkhauses platziert.
Die Aufnahmen werden mit Objekt-Erkennung analysiert.
Dabei werden lediglich Autos betrachtet, welche in das Parkhaus hinein bzw.\ aus dem Parkhaus hinaus fahren.
Durch die Objekterkennung kann ausgeschlossen werden, dass ungewünschte Objekte oder Personen vom System erfasst werden.

Die Parkplätze der Mitarbeiter müssen separat gezählt werden.
Eine Möglichkeit wäre die Aktivität der Schranke mit einem Neigungssensor zu erfassen.
Es kann jedoch potenziell dazu kommen, dass eine Schranke geöffnet wird und kein Auto durchfährt oder mehrere Autos gleichzeitig durch die geöffnete Schranke fahren.
Deshalb wird für die Zählung der Auslastung der Mitarbeiterparkplätze dieselbe Variante verwendet, welche für die Erfassung der Studentenparkplätze vorgeschlagen wurde.
