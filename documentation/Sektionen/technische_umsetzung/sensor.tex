\selectlanguage{ngerman}
\subsection{Sensor}\label{ch:Umsetzung_Sensor}
% TODO:
% - Objekterkennung auf Frames: vortrainierte Modelle
% - Schwierigkeit: Nicht nur Erkennung einzelner Objekte nötig, sondern Erkennung des Rein- und Rausfahrens der Autos
%     --> zwei Implementierungen, Verweis auf Tests

\subsubsection{Variante 1: Richtung des Bewegungsvektors}\label{ch:Sensor_v1}
% TODO:
% - Grundidee: Erkennen in welche Richtung im Bild (oben oder unten) Auto fährt
% - Vorgehen: 
%     - Rechtecke (Eckkoordinaten) der Autos durch Objekterkennung
%     - Berechnung der Mittelpunkte
%     - Berechnung der Bewegungsrichtung 
%         - Für jeden Punkt im aktuellen Frame werden die Distanzen zu den Punkten des vorherigen Frames errechnet
%         - Der Punkt mit der jeweils kürzesten Distanz wird ausgewählt und mit diesem der Bewegungsvektor errechnet. Falls bereits passender früherer Bewegungsvektor vorhanden, wird Aufaddiert, da nur Gesamtergebnis wichtig  
%         - Punkte des vorherigen Frames, die für keinen aktuellen Punkt die kürzeste Verbindung waren (weil jetzt weniger Autos im Bild), werden gesucht und deren Bewegungsvektor für Zählung verwendet 
%     - Anhand der Richtung der y Koordinate es Vektors die Richtung (oben/unten) des Autos bestimmen 
%     - API Aufruf

\subsubsection{Variante 2: Überschreiten einer Linie}\label{ch:Sensor_v2}
% TODO:
